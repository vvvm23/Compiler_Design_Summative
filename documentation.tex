\documentclass[11pt,a4paper]{article}
\usepackage[margin=1in]{geometry} 
\usepackage[T1]{fontenc}
\usepackage{lmodern}

\begin{document}
\title{Software Methodologies: Compiler Design Summative}
\author{vvvm23}
\date{}
\maketitle
\hrule

\section{Running the Program}

\subsection{Prerequisites}

The program requires the following prerequisites:

\begin{itemize}
    \item Python 3.7.4 must be installed.
    \item graphviz must be installed. Package managers should have this but here is a gitlab link: https://gitlab.com/graphviz/graphviz/
    \item Lab computers should have all packages except for pygraphviz which should be installed.
\end{itemize}

\subsection{Command Line Options}

The program has the following positional command line arguments:
\begin{itemize}
    \item input\_path: Path to input file
    \item log\_path: Path to the logging file. If not specified, defaults to log.txt
\end{itemize}
Run the program by running either:
\begin{itemize}
    \item python submission.py input.txt log.txt
    \item python submission.py input.txt
\end{itemize}
\\

\hrule

\section{Output Files}
\subsection{Productions}
If the input file is valid the production rules for the grammar will be printed to the console (not the log file). The starting symbol is form.

\subsection{Log file}

If the input file was not valid this will be reflected in the console. There will be no changes to the log file.

If the input file was valid and the formula was well formed, the program will show this in the console and write a similar success message in the log file.

If the input file was valid but the formula has a syntax error, the program will display the error in the console and also display the error in the log file.

Errors while parsing are formatted as such:
\begin{enumerate}
    \item Notify there is a syntax error and at what position.
    \item Display the formula with the error highlighted in colour (in console) or surrounded by > > > < < < (log file).
    \item Specific Error Code and explanation of it.
    \item Additional information if available, such as a suggestion on corrections.
\end{enumerate}

\subsection{Parse Tree}

If the input file was not valid or the formula in it was not valid, no parse tree will be saved to file.

If the formula is well formed, the parse tree will be saved to tree.png

tree.png will display the input formula at the top and below it will be the parse tree for the formula.

\section{Error Codes}
The following describes the error codes that may be encountered when parsing a valid file:
\begin{itemize}
    \item GENERIC - The Code produced when no other code was set.
    \item UNKNOWN\_SYMBOL -
    \item UNEX\_SYMBOL -
    \item EX\_VAR - 
    \item EX\_VC -
    \item EX\_EQ -
    \item EX\_CONN2 -
    \item EX\_BRACKET -
    \item EX\_COMMA -
    \item EX\_END - 
\end{itemize}

\end{document}

