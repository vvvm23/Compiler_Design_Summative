\documentclass[11pt,a4paper]{article}
\usepackage[margin=1in]{geometry} 
\usepackage[T1]{fontenc}
\usepackage{lmodern}

\begin{document}
\title{Software Methodologies: Compiler Design Summative}
\author{vvvm23}
\date{}
\maketitle
\hrule

\section{Running the Program}

\subsection{Prerequisites}

The program requires the following prerequisites:

\begin{itemize}
    \item Python 3.7.4 must be installed.
    \item graphviz must be installed. A windows MSI can be found here https://graphviz.gitlab.io/download/
    \item Lab computers should have all packages except for pygraphviz which should be installed.
\end{itemize}

\subsection{Command Line Options}

The program has the following positional command line arguments:
\begin{itemize}
    \item input\_path: Path to input file
    \item log\_path: Path to the logging file. If not specified, defaults to log.txt
\end{itemize}
Run the program by running either:
\begin{itemize}
    \item python submission.py input.txt log.txt
    \item python submission.py input.txt
\end{itemize}
\\

\hrule

\section{Output Files}
\subsection{Productions}
If the input file is valid the production rules for the grammar will be printed to the console (not the log file). The starting symbol is form.

\subsection{Log file}

If the input file was not valid this will be reflected in the console. There will be no changes to the log file.

If the input file was valid and the formula was well formed, the program will show this in the console and write a similar success message in the log file.

If the input file was valid but the formula has a syntax error, the program will display the error in the console and also display the error in the log file.

Errors while parsing are formatted as such:
\begin{enumerate}
    \item Notify there is a syntax error and at what position.
    \item Display the formula with the error highlighted in colour (in console) or surrounded by > > > < < < (log file).
    \item Specific Error Code and explanation of it.
    \item Additional information if available, such as a suggestion on corrections.
\end{enumerate}

\subsection{Parse Tree}

If the input file was not valid or the formula in it was not valid, no parse tree will be saved to file.

If the formula is well formed, the parse tree will be saved to tree.png

tree.png will display the input formula at the top and below it will be the parse tree for the formula.

\section{Error Codes}
The following describes the error codes that may be encountered when parsing a valid file:
\begin{itemize}
    \item GENERIC - The Code produced when no other code was set.
    \item EMPTY - The input string was empty. By the rules of the grammar this is an invalid string.
    \item UNKNOWN\_SYMBOL - If the symbol encountered was not defined in one of the fields in the input file (other than formula)
    \item UNEX\_SYMBOL - The symbol encountered was not expected to appear at this position. In these cases we cannot determine what should have been here easily.
    \item EX\_VAR - The parser expected a variable to be in this position, but the lookahead symbol is not one of the symbols recognised as a variable. Typically this will be associated with quantifier and predicate expressions.
    \item EX\_VC - The parser expected either a variable or a constant at this position, but the lookahead symbol is not a variable or a constant. This will be encountered when evaluating the second argument of an equality.
    \item EX\_EQ - The parser expected an equality symbol at this position, but the lookahead symbol is not one of the symbols recognised as an equality. This error will be thrown if an opening bracket followed by either a variable or a constant, but not preceded by a predicate, is found, but no equality symbol is found after.
    \item EX\_CONN2 - The parser expected a connective with arity 2 at this position, but the lookahead symbol is not one of the symbols recognised as a connective with arity 2. This error will be raised if one formula within brackets is evaluated successfully, but there is no connective to join it with the second expected formula.
    \item EX\_BRACKET - The parser expected either an opening or closing bracket at this position, but the lookahead symbol is not either of these. There are many places where this error could be thrown.
    \item EX\_COMMA - The parser expected a comma at this position, but the lookahead symbol is not a comma. This will occur if the parser expected additional arguments in a predicate, or if the arguments have not been separated by commas.
    \item EX\_END - The parser expected the end of the string but there are still more symbols in the string to process. This occurs if the parser evaluates all rules so far successfully and terminates, but the parser index is not equal to the length of the input string.
\end{itemize}

\end{document}

